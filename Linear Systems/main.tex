\documentclass[11pt,fleqn]{article}

\include{exercises}
\include{math_commands}
\usepackage{tcolorbox}

\Name{Claudia Hyun, Arsh Zahed}
\Class{Worksheet 1}
\Assignment{Math 21b}
%\Date{Date here}

\begin{document}

\begin{center}
\begin{tcolorbox}[title={\fontsize{13pt}{13pt}\selectfont Introduction to Linear Systems}, enforce breakable, pad at break=4mm, colframe=darkcrimson!100,colback=black!1,width=.9\linewidth,sharp corners=north]\fontsize{11.5pt}{12pt}\selectfont 
A \textbf{linear equation} of the variables $x_1, \hdots, x_n$ is an equation that can be written of the form 
$$
a_1 x_1 + a_2 x_2 + \hdots + a_n x_n = b
$$
where $a_1, \hdots, a_n$ and $b$ are real or complex numbers, and n may be any positive integer. For example, the equation $3x = 6$ is a linear equation. Using basic algebra, you should be able to solve the equation to find that $x=2$. \\
Similarly, you can create a \textbf{system of linear equations}, or a \textbf{linear system}. A linear system is a collection of linear equations with the same variables, such as 
$$
2x_1 + 3x_2 = 1
$$ 
$$
3x_1 - x_2 = 7
$$ 
We can also represent the linear system with a \textbf{matrix} to make more sense of the equations. We place the coefficients of the above system in a two rows and two columns of a matrix, unsurprisingly called a \textbf{coefficient matrix}, as follows 
$$\left[
\begin{array}{cc}
2 & 3 \\
3 & -1 \\
\end{array}
\right]$$
where the first column corresponds to $x_1$ and the second column with $x_2$. This isn't enough to find a solution though; we need the constant terms as well. We can append these in an \textbf{augmented matrix} as follows
$$\left[
\begin{array}{cc|c}
2 & 3 & 1 \\
3 & -1 & 7 \\
\end{array}
\right]$$
where the column after the line represents the constant terms. 
\end{tcolorbox}

\begin{tcolorbox}[title={\fontsize{13pt}{13pt}\selectfont Solving Linear Systems},enforce breakable, pad at break=4mm, colframe=darkcrimson!100,colback=black!1,width=.9\linewidth,sharp corners=north]\fontsize{11.5pt}{12pt}\selectfont
A \textbf{solution} for the linear system is a combination of values for the variables that satisfies all the equations in the system. A linear system can fall under three categories.
\begin{enumerate}
\item One solution
\item Infinitely many solutions
\item No solution
\end{enumerate}
If the system has one or infinitely many solutions, it is said to be \textbf{consistent}. If it has no solution, it is \textbf{inconsistent}.
\blni

You may remember from Algebra that linear systems can be solved using methods such as substitution and graphing; however, we can use matrices to make solving linear systems easier. Before developing an algorithm for solving linear systems with matrices, let's address some specific rules and scenarios. \\
When solving a matrix, the following procedures will not alter the solution.
\begin{enumerate}
\item Interchanging rows \\
\item Scaling rows (multiplying by a nonzero constant) \\
\item Adding/subtracting one row to/from another \\
\end{enumerate}
The process of using these rules to solve matrices is called \textbf{Gauss-Jordan Elimination}. Using these rules, we can simplify matrices till a solution is apparent. For example, consider the following example.
$$\left[
\begin{array}{ccc|c}
1 & 2 & 0 & 7 \\
0 & 2 & 0 & 4 \\
0 & 1 & 1 & 6 \\
\end{array}
\right]$$
We can subtract the second row from the first row and then scale the second row by a factor of $\frac{1}{2}$ to get reduce the matrix to
$$\left[
\begin{array}{ccc|c}
1 & 0 & 0 & 3 \\
0 & 1 & 0 & 2 \\
0 & 1 & 1 & 6 \\
\end{array}
\right]$$
We can now subtract the second row from the third row.
$$\left[
\begin{array}{ccc|c}
1 & 0 & 0 & 3 \\
0 & 1 & 0 & 2 \\
0 & 0 & 1 & 4 \\
\end{array}
\right]$$
Because all the nonzero coefficients are on the \textbf{diagonal} of the matrix, it is immediately clear that $x_1 = 3$, $x_2 = 2$ and $x_3 = 4$, giving us our solution.\\
The last matrix, with all the nonzero terms on the diagonal, is in \textbf{reduced row-echelon} form. This is the goal when reducing matrices. The algorithm for reducing matrices to reduced row-echelon form is given in the lecture notes, but is duplicated here for convenience.\\

\begin{tcolorbox}[title={\fontsize{13pt}{13pt}\selectfont Strategy for getting to the reduced row-echelon form},enforce breakable, pad at break=4mm, colframe=crimsom!80,colback=black!1,width=.9\linewidth,sharp corners=north]\fontsize{11.5pt}{12pt}\selectfont
\begin{enumerate}

\item Scale or interchange to produce a Leading 1 in the upper left position (if possible). A Leading 1 refers to

where the first nonzero entry on a row (as read from left to right) is equal to 1.

\item Use the first row as a pivot row, leaving it unchanged but adding (or subtracting) appropriate multiples of it

to the other rows to produce 0s elsewhere in the column contain the Leading 1 (the first column in this

case). We call this cleaning the column.

\item Scale or interchange to produce a Leading 1 in the second row ? shifted one column to the right (possibly

more).

\item Using the second row as the pivot row to clean this next column leaving only the Leading 1 in the pivot row

and 0s elsewhere in that column.

\item Continue this process to get Leading 1s in each of the rows, shifting to the right as you descend through the

rows.

\item Any all-zero rows should appear at the bottom of the array.
\end{enumerate}
\end{tcolorbox}

\end{tcolorbox}

\end{center}

\end{document}